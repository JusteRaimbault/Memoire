

\documentclass[english]{article}
\usepackage[T1]{fontenc}
\usepackage[utf8]{luainputenc}
\usepackage{graphicx}

\makeatletter

\newcommand{\noun}[1]{\textsc{#1}}

\usepackage{background}


\SetBgOpacity{0.3}
\SetBgColor{black!20}

\setcounter{page}{50}

\makeatother

\usepackage{babel}
\begin{document}

\title{Paper D\\
Local and global design of a new district through evolutionary algorithm}

\maketitle
\vspace{2cm}

\begin{abstract}
The study of refurbishment of old building stocks through Agent-based
Modeling in Sweden (?) has occulted on purpose the background problem of a parallel
building of a totally new district (Project Homes for Tomorrow)
which could have strong influence on the existing district. Although the
design has already been done and the construction process is well
advanced, we propose here to follow the experience of suggesting
a possible design from scratch through computational methods for this
new district. We explore first the question of the role of computational
design in architecture and the one of multi-objective optimisation
in urban planning, and how it could be interesting to link them in
order to construct an hybrid approach. Then we describe a simplified
algorithm that we apply to our concrete problem to propose automatic
design, and that we finally couple with the Agent-Based Model for
the existing District.

\newpage{}
\end{abstract}

\section*{Introduction}


\section{Overall approach of the problem}


\subsection{From Artificial Art to Computational Design, applications in Architecture}

The use of softwares and computers to produce Art, what we can call
artificial art, has recently developped as a branch of practical applications
of computational algorithms.

\bigskip{}


The designer \noun{Casey Reas} is one of the pioneers in that field
and is still very active today. He formulates his vision on the use
of software in design in \cite{reas2010form+}, that can be considered
as his manifesto. As an interpretation of his ideas, we suggest that
imagination can be seen as a powerful method of exploration of the
space of possibilities (which we paradoxaly present as the search
space of the mind, but which is not rigourously defined in that case).
It appeared quickly as an application of artificial intelligence
to simulate artifacts of the imagination, with its own advantages
and issues. Typically, by using algorithms to create pieces of art
or to design shapes, we just suggest new ideas to help the mind to
create, but the process is still entirely guided by the artist. The
software is then only a new medium of expression and exploration of
artistic possibilities.

\noun{Reas} has explored figurative and abstract software painting,
but also applications in design and architecture. One example of his
work in architectural design can be seen in figure 1 .

\begin{figure}


\includegraphics[scale=0.5]{images/PaperD/casey-reas-the-eye-of-the-needle}\caption{Software as an expression medium (source : \cite{reas2010form+})}


\end{figure}


\bigskip{}


The most recent research in artificial art pushes towards the use of artificial
ants. In \cite{monmarche2008artificial}, the authors, that are specialists
of solving complex optimisation problems using systems of artificial
ants, present how these ``ants'' can be used to produce paintings
or music. In a forthcoming work (presented at \cite{lectureAnts}),
\noun{Monmarché} proposes some applications to the design of real
forms in three dimensions, first for everyday objects, but also for
sculptures. Following local rules (as for the 2D systems) and global
constraints (it is one of the limitations of the method that still
need to be overriden), the ants build objects and shapes in space.
It could be easely applied to design of architectural shapes. An exemple
of abstract art obtained by ants is shown on figure 2.

\begin{figure}


\includegraphics[scale=0.5]{images/PaperD/artAnts}\caption{Ant Painting (source : \cite{monmarche2008artificial})}


\end{figure}


\bigskip{}


An other important aspect of mathematical methods applied to design
is the use of shape grammars in computational design. \noun{T. Knight}
has recently proposed a theorization for the use of these methods
; in \cite{knight2003either}. In what is more an essay than an
paper with a definite purpose, he tries to characterise the ideas behind this type of art.
He manages to make a link between a generative rule (the title of
the paper, ``either, or -> and'') for computational algorithms with
an implicit spirit of Bauhaus' artists like \noun{Kandinski} or \noun{Klee}.
The rule is applied to many concepts of art philosophy in order to
show that the Bauhaus artistic movement followed this abstract rule
in many ways. Without going into details, one of the most interesting
application (for our purpose) is the link between emergence and predictability,
since it meets epistemological notions that are crucial in our work.
At the heart of the study of complex system, emergence can be seen
in a simplified way as the global properties of the system that come
from the interaction of its local parts and that are by essence not
predictable (it can be argued that this notion is really more subtle,
as it is defined in different ways by \noun{Bedau} in \cite{bedau2002downward},
but we will stick to this simple definition for our problem). Therefore,
it is difficult to combine emergence and predictability, but still,
these artists have somehow managed that. As an example, for \noun{Knight}, ``a
good design requires both emergence and predictability''. That is
a real issue when dealing with shape grammars : one can create incredible
shapes, but they have to follow given constraints (so to show predictability)
to be applied to architectural problems. A concrete application of
this theory is proposed in \cite{knight2003computing}, where he shows
examples of shape generation through the application of shape grammar
(formally it is a grammar which rule can evolve according to user
constraints, which rules apply to subsets of the plan). Further developments
have been proposed recently, as the application on curved shapes in
\cite{jowers2010construction}, concretely implemented through Bezier's
curves calculations in \cite{jowers2011implementation}, allowing
new designs including curves. The principles of shape grammars are
shown in figure 3.


\subsection{Evolutionary Algorithms for multi-objective Optimisation}

On a greater scale, algorithms are also used to explore a space of
possibilities, not for design purposes but to solve optimisation problems.
Whereas simple objective optimisation can be easily solved by classic
methods as the gradient descent as soon as some hypotheses on the
function are met, the multi-objective optimisation problems present
instrinsic difficulties to their resolution. In an analog way as combinatory
problems on great cardinal sets, an exhaustive exploration of the
definition set of the function is most of the times not possible with
the current technical means, what imposes the need of performants
exploration algorithms.


\subsection{Mixing the two approaches : heterogeneous modeling and design through
scale integration}

We propose here an approach to the problem of multi-scale urban planning
and design that could be considered as a mix of the two techniques
described in the preceeding sections. Indeed, it appears that they
may be not so far away from each other and that the gap would just be a question of scale.
To go further, we can try to make some hypotheses that of course would
need deeper investigation. Yet they can offer strong potential for future
work : maybe the frontier between architecture and urban design is not
so clearly defined. A mere change of point of view or scale can break
this edge. Some evidences, as the existence of an ``architectural
urbanity'' that we have presented in cite?, or the parallel done above
between computational methods at different scales, can reinforce that vision.

\bigskip{}


Therefore we will try here to explore the possibilities offered by
that point of view by building a scale-integrated model for automatic
design : in a really simple way (we do not try to elaborate complicated
model, the aim is more a beginning of investigation of the fundamental
ideas). We will couple a macro-scale evolutionnary algorithm for the
design of land use and transportation network with a local shape-generation
algorithm for the buildings. The main idea that we use for scale coupling
will be closer to indirect feedback through sub-systems interactions inside
the global system, as it was proposed for multi-scale cellular automata
coupling in \cite{MultiScaleCA10}, than to a direct feedback on the variable
and parameters at both scale. Indeed, it is done between macro differential
equation and micro agent-based model for evolution of individuals
in the work of Duboz on marine ecology in his thesis (\cite{Duboz:phd}).


\section{Designing the new Kvillbäcken through evolutionary computation}


\subsection{Preliminary work and overall assumptions}


\subsection{Formal description of the algorithm}


\subsubsection{Framework}

The space is discretised in a set of patches $\mathcal{P}=\{p_{0},...,p_{n}\}$.
We will distinguish the description of landuses at the macro level,
that, given a number of distincts landuses $N$, can be translated
through a function $L:\mathcal{P}\rightarrow\left[\left|1;N\right|\right]^{n}$,
and the micro description of the shapes of the buildings, that would
be for a configuration a family of continuous parts of surfaces in
space $ $$(S_{i})_{1\leq i\leq K}$.


\subsubsection{Optimisation objectives}


\subsection{Exploration and application}


\subsubsection{Patterns for landuse}


\section{Coupling the model with the ABM for Langängen}


\section*{Conclusion}

\newpage{}

\bibliographystyle{plain}
\bibliography{/Users/Juste/Documents/ComplexSystems/Biblio/BibTeX/global}

\end{document}
