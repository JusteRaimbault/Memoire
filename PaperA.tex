\documentclass[english]{article}
\usepackage[T1]{fontenc}
\usepackage[utf8]{luainputenc}
\usepackage{babel}
\begin{document}

\title{Architectural and sociological analysis of the Swedish peri-urban
areas : past, present and perspectives}
\maketitle
\begin{abstract}
We proceed to a description of the Swedish peri-urban areas. First
we describe the history of their development and of their
architectural features. Then a sociological diagnosis is presented
around the existence of the Swedish ``suburb'', and finally we review
some perspectives on the future of these areas.
\end{abstract}

\section*{Introduction}


\section{Architectural history}


\subsection{People's Home : Modern architecture for social housing}


\subsection{Million Homes Program : the Welfare State at a greater scale}

After 1960 began an other area in the implication of the State in
dwelling buildings, since a severe lack of the global quantity of
dwellings was constated. After the war, Sweden followed a quick urbanization
process, but the infrastructure was not ready to face such an increase
in demand.


\section{``The myth of the Swedish suburb''}


\section{Perspectives}

\bibliographystyle{plain}
\bibliography{/Users/Juste/Documents/ComplexSystems/Biblio/BibTeX/global}

\end{document}
