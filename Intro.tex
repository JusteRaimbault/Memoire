
\documentclass[english]{article}
\usepackage[T1]{fontenc}
\usepackage[utf8]{luainputenc}
\usepackage{graphicx}

\makeatletter

\newcommand{\noun}[1]{\textsc{#1}}

\date{}

\makeatother

\usepackage{babel}
\begin{document}

\title{Introduction}

\maketitle

\section{Setting the scene}

This project can be situated at the intersection of multiples perspectives
and approaches on architecture and urban design and follows therefore
the actual trend of a more integrated and transversal way to practice
architectural science. Indeed, the bottom-up systemic approach is
quite recent in urban design (introduced for example by the architect
\noun{Batty} in the end of the 90s, recently sumed up in \cite{Bat07}),
and more recent for architecture in itself (for exemple first papers
on the use of computational design, as the work of \noun{Knight} on
shape grammars in \cite{knight2003computing}, are not older than
ten years), although earlier work already proposed implicitly top-down
systemic for urban systems, for which the best example is the theory
of Space Syntax introduced by \noun{Hillier} in 1976 in \cite{hillier1976space}.
More than the theorical already present, it is really the technical
aspect of complex systems that is breaking nowadays into design and
planning, as a consequence of the recent development of complexity
analysis in link with new accessible and always more powerful computer
simulation tools (see \cite{chavalarias2009french} to have an idea
of the development of that disciplin).

\bigskip{}


The way to practice and to think architecture really depends of the
cultural context. As an example, a friend studying nowadays in a school
of architecture in France explained his vision of the disciplin :
``there are three ways to do Architecture, the technical one {[}he
meant construction technique{]}, the artistic one and the social science
one''. Let take this as an unformal description, since we weren't
able to find a similar description in the litterature, but that should
reflect a ``field'' reality. Compared to that, the practice of research
in Architecture in Sweden seems to be more oriented towards social
science, and the systemic vision has its place as a branch of research
in Architecture and Urban Planning ; in France, the field is not so
developped and is studied by geographers. Our purpose is to claim
that our study may be at the border of several disciplines and at
the cross of points of views that may appear as opposed, but we bet
on the fact that this originality could bring interessant ideas.

\bigskip{}


The implicit personal opinion behind the philosophy of this work is
that the gap that could exist between the artistic aspect and the
scientific aspect (that can be social science, technical aspect or
complex systems science) has no strong justification and that it could
be breached through epistemological, philosophical and technical work
- a concrete example of such a try will be presented in the last paper
composing this project, when we will propose an hybrid model based
on scale integration. Therefore we will always try to keep a multidisciplinary
point of view when posing the problematics and proposing solutions.

\bigskip{}


The general purpose of this work, as presented in the initial proposal
(see \cite{RaimbaultReBoDescription0412}), is to propose extensions
of technical order to an existing research project lead at the departement
of Architecture at Chalmers University of Technology, Göteborg, by
applying in priority agent-based modeling, and more generally evidence-based
methods. We use the theorical framework of the project (that we present
in more details in the following) to build models which application
is supposed to strengthen the ideas proposed by it.


\section{Integrating sustainable processes : the project ReBo}


\paragraph{General presentation}

The need of integrated processes has appeared as crucial for the sustainibility
of the projects, and in that frame the consideration of refurbishment
or requalification of buildings has totally its place. First the refurbishment
in itself becomes sometimes necessary to fill ecological and societal
constraints of sustainibility, but reciproquely a refurbishment can
not be sustainable if its approach is not integrated. For example,
strong cultural and architectural (in the sense of the subjective
quality) components can become an issue, then an asset for sustainibility
in a refurbishment process. These ideas have been first formulated
by \noun{Stenberg} \& al. in \cite{stenberg2009linking}, where they
insist on the importance of taking into account societal aspects (will
the refurbished district be ``socially sustainable''?) and many
other fields within the design of a refurbishment process. They point
the fact that ``Actually, it is not unusual that »environmental projects«
in {[}...{]} areas are put forward as examples of holistic and good
practice in urban development in general, without considering fundamental
and negative social impacts'', and therefore conclude to a huge need
of transversal integration in the processes. The general transversal
integration is shown on figure 1.

\begin{figure}
\includegraphics[scale=0.32]{images/Intro/rebo}\caption{Multiples aspects taken into account in the evaluation by ReBo framework.}


\end{figure}


\bigskip{}


This has lead to the development of a research project called ReBo
conducted since 2010 by Chalmers University of Technology in Göteborg,
in which many other institutionnals stakeholders (komun, firms) are
sometimes involved. The aim is to propose a multi-objective evaluation
framework for sustainable refurbishment processes, that could be easily
used by tenants and parts involved. In \cite{thuvander2011strategies},
\noun{Thuvander} \& al. describe the theorical framework, built from
the concrete case of Swedish housing stocks from a particular type.
They insist on the transversal integration (differents aspects taken
into account as technical description, environmental performance,
social aspects, cultural aspects, architectural legacy, etc. ) but
also on the vertical integration, by giving for each parameter what
they call ``parameter levels'', that are the different scale levels
for evaluation : the deepest is the local concrete observation of
real proxys for the considered parameter, the highest the global notion
around the parameter.


\paragraph{Objectives and means}

We described the abstract aim of the project but didn't give any concrete
objective. They seem to have appeared as consequences of the theoretical
research, and one of the best example is the concrete establishment
of rigourous framework and methods at all step of the refurbishment
process. Therefore, a further step of the project was the holding
of workshops with all types of stakeholders to identify the current
practices and tools used for evaluation of refurbishment need and
in the process itself.


\section{The question of modeling}

Since we want to propose formal models on which simulations can be
launched and that could lead to evidence-based solutions to the problematics
we face, and since a purely technical approach is not the purpose
of our work, we were strongly confronted to the epistemological issue
of the sense and the role of modeling.


\paragraph{Modeling in our context}


\paragraph{Why, what and how proceed to modeling?}


\section{Description of the project}


\paragraph{Overview}


\paragraph{Plan of work}

\bibliographystyle{plain}
\bibliography{/Users/Juste/Documents/ComplexSystems/Biblio/BibTeX/global}

\end{document}
