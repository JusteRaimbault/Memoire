\documentclass[english]{article}
\usepackage[T1]{fontenc}
\usepackage[utf8]{luainputenc}
\usepackage{graphicx}

\makeatletter

\newcommand{\noun}[1]{\textsc{#1}}

\date{}

\makeatother

\usepackage{babel}
\begin{document}

\title{Introduction}

\maketitle

\section{Setting the scene}

This project stands at the intersection of multiple perspectives
and approaches on architecture and urban design. It follows 
the current trend of a more integrated and transversal way to practice
architectural science. Indeed, this bottom-up systemic approach is
quite recent in urban design (introduced for example by the architect
\noun{Batty} at the end of the 90s, recently summed up in \cite{Bat07})
and (even?)more recent as far as architecture in itself is concerned (for exemple first papers
on the use of computational design, as the work of \noun{Knight} on
shape grammars in \cite{knight2003computing}, are not older than
ten years). However,earlier work already proposed implicitly top-down
systemic (approach?) for urban systems, for which the best example is the theory
of Space Syntax introduced by \noun{Hillier} in 1976 in \cite{hillier1976space}.
As the theorical component is already well-established, it is really the technical
aspect of complex systems that is breaking nowadays into design and
planning. This is a consequence of the recent development of complexity
analysis in link with always more powerful computer
simulation tools (see \cite{chavalarias2009french} to have an overview
of the development of that disciplin).

\bigskip{}


The way to practice and to reflect on architecture really depends of the
cultural context. As an example, a friend studying nowadays in a school
of architecture in France explained his vision of the disciplin :
``there are three ways to do Architecture, the technical one {[}he
meant construction technique{]}, the artistic one and the social science
one''. Let us take this as an unformal description, as it should
reflect a ``field'' reality, even though it has not been formalised by the literature. Compared to that, the practice of research
in architecture in Sweden seems to be more oriented towards social
science. The systemic vision has its place as a branch of research
in Architecture and Urban Planning ; in France, the field is not so
developped and is mainly studied by geographers. Our purpose is to claim
that our study may be located at the frontier between several disciplines and at
the crossing of points of views that may appear as opposed. We claim that this original approach could bring ideas of interest.

\bigskip{}


The implicit personal opinion behind the philosophy of this work is
that the gap that could exist between the artistic aspect and the
scientific aspect (either social science, technical aspect or
complex systems science) has no strong justification. It could
be breached (breached?) through epistemological, philosophical and technical work
- a concrete example of such an attempt will be presented in the last paper
composing this project, in which we will propose an hybrid model based
on scale integration. Therefore we will always try to keep a multidisciplinary
point of view when exposing the hypotheses and proposing solutions.

\bigskip{}


The general purpose of this work, as presented in the initial proposal
(see \cite{RaimbaultReBoDescription0412}), is to propose extensions
of technical order to an existing research project led at the departement
of Architecture at Chalmers University of Technology, Göteborg. This work will be
applying in priority agent-based modeling and more generally evidence-based
methods. We use the theorical framework of the project (that we present
in more details in the following section) to build models of which application
is supposed to strengthen the ideas they offer.


\section{Integrating sustainable processes : the project ReBo}


\paragraph{General presentation}

The need of integrated processes has appeared to be crucial for the sustainability
of the projects and the consideration of refurbishment or requalification of buildings fits in this framework. First the refurbishment
in itself becomes sometimes necessary to fulfil ecological and societal
aspects (constraints est un peu négatif) of sustainability. Reciprocally a refurbishment can
not be sustainable if its approach is not integrated (You may want to rephrase that sentence!). For example,
strong cultural and architectural (in the sense of the subjective
quality) components can become an issue, then an asset for sustainibility
in a refurbishment process (this one as well). These ideas have been first formulated
by \noun{Stenberg} \& al. in \cite{stenberg2009linking}, where they
insist on the importance of taking into account societal aspects (will
the refurbished district be ``socially sustainable''?) and many
other fields within the design of a refurbishment process. They point out
the fact that ``Actually, it is not unusual that »environmental projects«
in {[}...{]} areas are put forward as examples of holistic and good
practice in urban development in general, without considering fundamental
and negative social impacts''. Therefore they conclude to a huge need
of transversal integration in the processes. The general transversal
integration is shown on figure 1.

\begin{figure}
\includegraphics[scale=0.32]{images/Intro/rebo}\caption{Multiples aspects taken into account in the evaluation by ReBo framework.}


\end{figure}


\bigskip{}


This has lead to the development of a research project called ReBo that has been
conducted since 2010 by Chalmers University of Technology in Göteborg,
in which many other institutionnals stakeholders (komun, firms) are
sometimes involved. The aim is to propose a multi-objective evaluation
framework for sustainable refurbishment processes, that could be easily
used by tenants and parts involved. In \cite{thuvander2011strategies},
\noun{Thuvander} \& al. describe the theorical framework, built from
the concrete case of Swedish housing stocks from a particular type.
They (Thuvander est pas tout seul?) insist on the transversal integration (differents aspects taken
into account as technical description, environmental performance,
social aspects, cultural aspects, architectural legacy, etc... ) but
also on the vertical integration. They give to each parameter what
they call ``parameter levels'', that are the different scale levels
for evaluation : the deepest is the local concrete observation of
real proxys for the considered parameter, the highest the global notion
around the parameter.


\paragraph{Objectives and means}

We described the abstract aim of the project so now we can discuss its concrete
objectives. They seem to have appeared as consequences of the theoretical
research and one of the best example is the establishment
of rigourous framework and methods at each step of the refurbishment
process. Therefore, a further step of the project is to attend
of workshops with all types of stakeholders to identify the current
practices and tools used for evaluation of refurbishment need and
in the process itself.


\section{The question of modeling}

We want to design formal models on which simulations can be
launched which could lead to evidence-based solutions to the problems
we face. Moreover, a purely technical approach is not the purpose
of our work. In the end, we are strongly confronted to the epistemological issue
of the sense and the role of modeling.


\paragraph{Modeling in our context}


\paragraph{Why, what and how to proceed to modeling?}


\section{Description of the project}


\paragraph{Overview}


\paragraph{Plan of work}

\bibliographystyle{plain}
\bibliography{/Users/Juste/Documents/ComplexSystems/Biblio/BibTeX/global}

\end{document}
