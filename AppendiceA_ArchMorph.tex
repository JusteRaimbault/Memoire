\documentclass[english]{article}
\usepackage[T1]{fontenc}
\usepackage[utf8]{luainputenc}
\usepackage{units}
\usepackage{amssymb}
\usepackage{graphicx}

\makeatletter

\newcommand{\noun}[1]{\textsc{#1}}
\newcommand{\lyxdot}{.}


\usepackage{bbm}

\usepackage{background}

\date{May $14^{th}$ 2013}




%manage background
\SetBgOpacity{0.3}
\SetBgColor{black!20}

\@ifundefined{showcaptionsetup}{}{%
 \PassOptionsToPackage{caption=false}{subfig}}
\usepackage{subfig}
\makeatother

\usepackage{babel}
\begin{document}

\title{Architectural Morphology\\
Investigative modeling and spatial analysis}

\maketitle
\bigskip{}
\bigskip{}
\bigskip{}


\textit{\large \hfill{}Public Research Workshop\hfill{}\hfill{}}{\large \par}

\textit{\large \hfill{}Stockholm, KTH School of Architecture\hfill{}\hfill{}}{\large \par}

\bigskip{}
\bigskip{}
\bigskip{}
\bigskip{}

\begin{abstract}
The development of new theorical and technical means, particularly
in the field of computer science and its direct applications, leads
gradually to a renewal of the approach on design and architecture.
The increasing place of modeling and calculations in the architectural
process confirms that Architecture lays on the interface, in this
case ambiguous, between art and science. This workshop aims to be
a presentation of the state of the art of actual research in the field
of spatial analysis applied to urban design, urban planning and architecture.
\end{abstract}
\newpage{}

\textit{\large \hfill{}}\textbf{\Large Speakers}\textit{\large \hfill{}\hfill{}}{\large \par}

\bigskip{}
\bigskip{}


\textit{\large \hfill{}}\noun{John Peponis}\textit{\large \hfill{}\hfill{}}{\large \par}

\textit{\large \hfill{}}Professor, Associate Chair Advanced Studies
and Research\textit{\large \hfill{}\hfill{}}{\large \par}

\textit{\large \hfill{}}GeorgiaTech School of Architecture\textit{\large \hfill{}\hfill{}}{\large \par}

\bigskip{}


\textit{\large \hfill{}}\noun{Sophia Psarra}\textit{\large \hfill{}\hfill{}}{\large \par}

\textit{\large \hfill{}}Reader of Architecture and Spatial Design\textit{\large \hfill{}\hfill{}}{\large \par}

\textit{\large \hfill{}}The Bartlett School of Graduate Studies,
Editor, Journal of Space Syntax\textit{\large \hfill{}\hfill{}}{\large \par}

\bigskip{}


\textit{\large \hfill{}}\noun{Ermal Shpuza}\textit{\large \hfill{}\hfill{}}{\large \par}

\textit{\large \hfill{}}Associate Professor\textit{\large \hfill{}\hfill{}}{\large \par}

\textit{\large \hfill{}}Department of Architecture, Southern Polytechnic
State University\textit{\large \hfill{}\hfill{}}{\large \par}

\bigskip{}


\textit{\large \hfill{}}\noun{Meta Berghauser Pont}\textit{\large \hfill{}\hfill{}}{\large \par}

\textit{\large \hfill{}}Chair Urban Design – Theory and Methods\textit{\large \hfill{}\hfill{}}{\large \par}

\textit{\large \hfill{}}TU Delft Faculty of Architecture\textit{\large \hfill{}\hfill{}}{\large \par}

\bigskip{}


\textit{\large \hfill{}}\noun{Ulrika Karlsson}\textit{\large \hfill{}\hfill{}}{\large \par}

\textit{\large \hfill{}}Visiting Professor School of Architecture,
KTH; servo stockholm\textit{\large \hfill{}\hfill{}}{\large \par}

\bigskip{}


\textit{\large \hfill{}}\noun{Christian Deri}\textit{\large \hfill{}\hfill{}}{\large \par}

\textit{\large \hfill{}}Head of AEDAS Architects R\&D\textit{\large \hfill{}\hfill{}}{\large \par}

\textit{\large \hfill{}}Visiting Professor, Technical University
of Munich\textit{\large \hfill{}\hfill{}}{\large \par}

\bigskip{}


\textit{\large \hfill{}}\noun{Åsmund Izaki}\textit{\large \hfill{}\hfill{}}{\large \par}

\textit{\large \hfill{}}AEDAS Architects R\&D\textit{\large \hfill{}\hfill{}}{\large \par}

\bigskip{}


\textit{\large \hfill{}}\noun{Daniel Koch}\textit{\large \hfill{}\hfill{}}{\large \par}

\textit{\large \hfill{}}Researcher, Director of Research Studies\textit{\large \hfill{}\hfill{}}{\large \par}

\textit{\large \hfill{}}KTH School of Architecture\textit{\large \hfill{}\hfill{}}{\large \par}

\bigskip{}


\textit{\large \hfill{}}\noun{Pablo Miranda Carranza}\textit{\large \hfill{}\hfill{}}{\large \par}

\textit{\large \hfill{}}Researcher\textit{\large \hfill{}\hfill{}}{\large \par}

\textit{\large \hfill{}}KTH School of Architecture\textit{\large \hfill{}\hfill{}}{\large \par}

\newpage{}


\section*{Introduction\bigskip{}
\bigskip{}
}

\textit{This workshop was presented as follows :}

\bigskip{}


\bigskip{}


The Research Workshop Architectural Morphology: Investigative modeling
and spatial analysis is meant as a beginning or a point of departure,
in research and for coming events revolving around modeling and spatial
analysis in architecture. With speakers of considerable repute within
the field commonly referred to as Space Syntax, as well as in other
Architectural fields, it is meant to communicate cutting edge analytical,
configurative modeling as well as explore relations to other modeling
and analytical traditions in architectural research. Furthermore,
through the participation of AEDAS R\&D and the experience of many
of the speakers, the relation between modeling and analysis in research
and practice will be highlighted and discussed.

\newpage{}


\section*{\noun{John Peponis}, GeorgiaTech School of Architecture\protect \\
Concrete applications of Space Syntax}

\textit{We unfortunately missed the begining of this first presentation.}

\textit{That presented direct application of spatial integration calculations
on Atlanta districts.}

\bigskip{}


The theory of space syntax was introduced by \noun{Hillier} in 1976
in \cite{hillier1976space}. Its aim is to study the influence of
spatial configuration on human aspects of urban life. One current
implementation is the axial map extraction : we extract the axial
map of a place by considering linear spaces, in the sense of one feels
belonging to that space when evolving in it (it's globally what one
can see, that's why that leads to an axial map in the context of street
network analysis). Then we can built the topological graph corresponding
to the axial maps, and can calculate what we call ``spatial integration''
on it : with $N$ places and $d_{ij}$ the topological distance from
place $i$ to $j$, the mean accessibility to other places : $I_{S}=\frac{2}{N\cdot(N-1)}\cdot\sum_{i<j}d_{ij}$
, and the integration is the mean of all accessibilities on the graph.

\textit{The figure 1 shows that process of topological graph extraction.}

\bigskip{}


Concerning the concrete applications, it is possible for example,
by distinguishing pedestrian and car axial maps, to show evidences
of ``poorly'' designed district in the sense that they are not liveable
for a pedestrian, what is no more possible nowadays. Such designs
present a strong lack of flexibility.

\bigskip{}


Such errors could have been avoided by the use of analytic methods
like spatial analysis, so we need today to switch from an exclusive
descriptive approach of architecture to a normative point of view
; we need more normative, evidence-based practice. For example, back
to the Atlanta districts, it can be shown through investigative modeling
that a greater building flexibility could have been permitted thanks
to a non equivalent distribution of block size ; that can be put in
parallel with the need for local activity diversification.

\bigskip{}


Of course spatial analysis will never be the direct answer to the
difficult question of what is an ideal city, but the motivation of
space syntax has always been a normative aim through a better understanding
of urban systems.

\bigskip{}



\paragraph{Question}

\textit{Should not the designer adapt the used methods to the real
situation, in the sense that the normative means won't be the same
depending on the neighborhood?}

Yes of course. Here it is the ideas behind the methods that are important,
not the concrete implementation themselves. The ``normative'' is
more a systematic application of calculation and modeling in general
than a specific method of furmula. 

\bigskip{}



\paragraph{Question}

\textit{If Space Syntax would have existed 100 years ago, would it
have solved the problems of modern urban planning and changed the
vision of Le Corbusier for example?}

A theory is created within a particular context, but if it is consistent
and powerful then it can be applied to other situations and fields.
Therefore, because of the success of this theory today, it should
have worked the same in the past.

\begin{figure}
\includegraphics[scale=0.45]{\lyxdot \lyxdot /Workshop/syntax/PrincipeSyntax}

\caption{Topological graph extraction}
\end{figure}


\newpage{}


\section*{\noun{Sophia Psarra}, UCL\protect \\
The Venice variations : Interactions between generation and explanation}

\bigskip{}
\bigskip{}
\bigskip{}



\subsection*{1 Role of spatial analysis considerations in Design and Architectural
Knowledge}

Architecture and its narrative approach are direct consequences of
the geometric spatial configuration and an embodied experience, which
can be approximated through a topological description of space. That's
why geometry and topology play both key roles in architectural analysis
; they have in fact a strong relationship which form determine most
aspects of the architectural experience.

\bigskip{}


As a consequence, a useful tool of investigative modeling can be the
try of different geometric shapes associated with the same topology.
In that case, whereas the spatial integration stays the same (since
we define it in the classic way, as done in the first lecture) because
it depends only of the topological configuration, the visual integration
differs and is interesting to consider as a design criteria. The visual
integration can be defined as follows : the architectural structure
can be considered as a subset $A\subset P\subset\mathbb{R}^{2}$,
where $P$ is the part of the space we work in. Then the visual integration
of a point is the measure of the visually accessible subset taking
into account the architectural obstacles (walls). For $M\in P$, it
is defined as \[I_{v}(M)=\intop_{M'\in P \backslash S}{\mathbbm{1}_{\{M+t \vec{MM'}|t\in [0,1]\} \cap S = \emptyset} dS}\]

Such a criteria can also be generalised in 3 dimensions, by discrete
superposition of floor layers, or by an analog continuous definition.

\bigskip{}


Its use can then play role in the development of architectural knowledge.
In \cite{calvino1978invisible}, \noun{Calvino} explains that imagination
can be in fact considered as \textit{Ars combinatoria}, that means
finding one good configuration among all possibles. In other words,
creating is exploring the plurality of words. The knowledge can be
classified in 4 types : dialectic knowledge (empirical), encyclopedic
knowledge (to make predictions), analytic knowledge (calculations)
and creative knowledge (imagination) ; and design is in particular
the combination of the last two : it joins these two different types
of knowledge through functionnal aspect and the use of imagination.

\bigskip{}


To go further in the role of computation in design, we can consider
the work of Smithson in the 70s, and the concept of ``Mat-building''
developped particularly in \cite{smithson1974recognize}. The important
ideas are that architecture and urbanism are closely linked to the
notion of emergence, and that an only top-down approach is not sufficient,
a bottom-up approach is also needed, by considerations of evolutionnary
fields and local relations.

\bigskip{}


That importance of computers in design was later in the 90s confirmed
by the apparition of evolutionnal design, e. g. design through genetic
algorithm that use given rules of the genetic languages to compute
new designs. That is again a bottom-up approach for which the unpredictability
of the emerging properties is inherent to the system and its self-organisation.\bigskip{}



\subsection*{2 Evolution and Urban form : case Venice}

Venice can be seen as an archipel lago of monuments and open spaces.
It is interesting to study relations between spatial and visual integration
in it. The single consideration of pedestrian network is not enough
to understand the patterns in urban form.

\bigskip{}


That lead to the idea of proceeding to a network coupling between
pedestrian network and water network, since the canals are in Venice
as important as the streets, and they can be considered as streets
themselves. The coupling is done throug the locations of step access,
that allow to bank with boats.

\bigskip{}


The important results of that study is that the urban form were strongly
influenced by both networks, and that modelings taking into account
only one lead to weak correlations. To sum up, the evolution of Venice
was strongly determined by the coupling of its two networks.

\bigskip{}



\subsection*{3 Comparison to the project of hospital by Le Corbusier}

One major project by Le Corbusier at the end of his life was a porject
of huge modern hospital for Venice, that we can see on figure 2.

\begin{figure}
\hfill{}\includegraphics[scale=0.8]{../Workshop/venice/hospital.jpeg}\hfill{}\hfill{}\caption{Project for Venice hospital by le Corbusier}


\end{figure}


\bigskip{}


For Sarkis in \cite{sarkis2001corbusier}, the spirit of the project
is closely linked to the Mat-building. The hospital is like a city
in itself, and without going too far, we can make a parallel between
the coupling of the visibility network and the accessibility network
in the hospital and the street/water networks in the city of Venice.
Le Corbusier would have unconsciously understood the essence of the
city and built a project corresponding exactly to it, reproducing
the structure of Venice ?

\bigskip{}


The building follows scaling laws and the flexibility of the design
suggests a sort of functionnal optimisation. It has also both good
spatial and visual integrations. The analysis of the project by analytical
methods shows us hidden aspects and leads to a better understanding
of its internal mechanisms.\bigskip{}
\bigskip{}



\subsection*{Conclusion}

After having presented new applications and implementations of space
syntax theory, and having linked them to architectural history and
the sense of architectural knowledge, we have seen the rich opportunites
these methods offered both in urban planning analyis and architectural
project understanding.

\newpage{}


\section*{\noun{Daniel Koch}, KTH\protect \\
Architectural Interfaces \& Resilience}

\bigskip{}
\bigskip{}
\bigskip{}
\bigskip{}



\subsection*{Introduction}

Architecture can be seen as the interface between the one and other
: it has a strong impact on the social relations. Whereas socio-cultural
identification can explain differences in housings, on the opposite
how is architecture communicating with these socio-cultural considerations?

\bigskip{}



\subsection*{Characterization of interfaces}

We first need to characterize the interface we are responding to.
Without going into details, a study of the influence of typical architectural
objects on our presence and our movement, by a projection in a evaluation
plane for these two functions, emphasises the importance of open space.
People are connected through space, and maybe that's why we are studying
these connections. As a consequence the resilience would be the way
social statuts are defined regarding our vision of space.

\bigskip{}


The question is therefore to study how architecture relates to exterior,
but also how it interfaces with people inside. Architecture is in
itself in a way teh interface between arrangment of objects and movement
of people. The link between visual interface and physical accessibility
is in that frame interesting to look at ; the projection of configurations
in the plane visibility/accessibility gives a measure of assymetry.
These concrete calculations on existing buildings strengthen our knowledge
of this internal interface.

\bigskip{}


The role of the relation between interior and exterior has to be considered
since it has consequences on the calculation of the measures. A concrete
example is the existence of an external path that changes the accessibility
measures inside the building. Without the exterior, we are able to
explain better the relations and interfaces inside the building. In
that way, visibility measures explain better how the building is internally
built. An interesting result of investigative modeling through visibility
measures is the fact that fusion of spaces (``open space'') in a
building of offices strongly increases the social segregation (whereas
the common idea is that it would help proximity and decrease it).

\bigskip{}



\subsection*{Consequences for the resilience of the built environment}

Taking for definition of resilience the relations through space of
social agents in the built environment has strong consequences on
the results obtained through investigative modeling. Indeed, we need
to explore the sensitivity of interfaces regarding system parameters,
and to make tests of continuity of the response proxys before validating
these output functions. We need to compare the different measures
and look for the existence of strong discontinuities at given points,
since these discontinuities can bring conclusions to non-sense.

\bigskip{}
\bigskip{}


We might change the classification of configuration by defining the
continuity not on a measure of ``similarity'' {[}\textit{that appears
to be classic distance between geometric patterns through sums of
distances along local homeomorphisms and discrete adding/deletion
of shape parts}{]} but more a measure of ``seamness'', that would
be defined the reciprocal way, by declare as close in seamness configuration
close in resilience. Then the continuity will not be a problem anymore.
A path to explore in future work is the question of the existence
of an algorithm using discretization and reduction of configurations,
that could lead from geometrical similarity to this ``seamness'',
i. e. a class reduction algorithm that would not calculate explicitly
the quotienting function but find directly the class only through
spatial discretisation and simple geometrical reductions, since the
calculation of the function can in some cases be arbitrary sensitive
and lead to wrong results because of bord-effects of the implementation.
The existence of such an algorithm has no direct reason to be true,
and it is still an intuition now. But its proof would deeply change
calculation heuristics and results on the resilience of the built
environment.

\newpage{}


\section*{\noun{Pablo Miranda Carranza}, KTH\protect \\
Tools used nowadays in advanced spatial analysis}


\subsection*{Generalities}

In the previous presentation some results of spatial analysis were
presented but not in details the tools behind the calculations. We
show here example of these tools.


\paragraph{Examples}

The following list is not exhaustive and is just to give an idea of
the diversity of analytical methods used in spatial analysis.
\begin{itemize}
\item Random walk : for blind building or city exploration, Brownian motion
can be interesting to explore virtual public open space.
\item Graph analysis : for combinatorial problems, for example for generating
a configuration, the use of graph exploration and complex graph theory
can be a way to obtain efficient calculations.
\item Space syntax : the original space syntax through axial map is used
in our studies to understand the way we perceive space.
\item Tree clustering : By successively clustering the tree of spatial relations
in an office, we try to understand the social logic of space through
the position of sitting places of workers. The main issue is to divide
space into ``logical'' boxes. The following shows us an other way
to do it for other applications.
\end{itemize}

\subsection*{Analysis of bunker architecture through convex decomposition of space}

We can define an algorithm for extracting the convex decomposition
of an internal space, by associating a point to the closest walls.
That is equivalent to make lines parallels to the walls to go progressively
away from the walls until the space is totally filled. The vertexes
and the edges of the decomposition are point equidistants from several
walls. We can show that the convexity limits are maximal convex shapes
in the building. We must be aware that the decomposition and the results
depend on the definition of convexity wa have, for example the described
algorithm is not applicable with the definition of strict convexity,
but only large. The strict convex decomposition doesn't exist in general
in real building plans.

This method was applied to the analysis of bunker architecture. an
interesting application is to link that with the location of the emergency
exits and with the potential flows of people in the different spaces.
The particular case of bunker is one of quasi only functional architecture,
so the study of relation between shape and function is more relevant,
and the method we described here lead in that case to concluding results.

\newpage{}


\section*{\noun{Meta Berghauser Pont}, TU Delft\protect \\
Density, Architecture and the City}


\subsection*{Why study density?}

Through history, density of cities has always have a great importance.
As a concrete example, there is evidence of the link between a high
density of population and health problems in Ansterdam, Jordean at
the end of the 19th century. At the same time, regulations to constraint
the height of buildings according to the street width were taken all
over the world (see Paris of \noun{Haussman} for example). The promoters
of the Garden City took the aspect of a healthier city as a main argument.
In the late 50s, Jacobs proposed (\cite{Jac56}) in opposition to
these idealisms a return to a more natural and by consequence a more
dense city.

\bigskip{}


Today, density can still be an issue. Back to the example of Amsterdam,
the global density is too low, as a consequence of an explosion of
the urban footprint, and of different relative growths of land uses
(the proportion of dwellings went bigger).

\bigskip{}


We could try to give an answer to the question of arguments for or
against densification, but there are very much pertinent arguments
on both side, so the really important aspect that appears is the study
of density in itself, the fact that it has good or bad consequences
on some aspects of the urban system is in fact an other problem, depending
most of the time on the particular concrete situation we are in.


\subsection*{Measuring density}

There exists in the litterature several way of measuring density,
and each is particular to the specific defintion given to density
and the field of application of the results. For example, the physical
density is different from the perceived density, or the demographic
density.

\bigskip{}


One important issue of the classic measure is that they don't manage
to capture in a single way the urban from and other aspects of urban
life. For example, the floor space index, the ground space index or
the open space ratio are currently used measures with their advantages
as qualifying in a way the perceived space. But still, the urban form
is not captured, and the size of the elements is not included, as
we can see on concrete example (same density for radically different
forms).

\bigskip{}


By considering the network density, in a way the streets per area,
we believe being able to define density in a new way that would capture
urban form. This hypothesis of coupling network analysis with spatial
considerations is our current research work which will lead we hope
to a new way of considering density


\subsection*{Performance of density}

An other important aspect in the study of density is the measure of
the performance of density ; it is also the object of furhter development
in our research to try to express simplified relations between density
function and several qualities of the urban life, for example the
parking performance of a street (available places to park the car)
or the daylight performances of the buildings.

\bigskip{}


We need to mix several aspects in the expression of the performance
of density. One issue would be to understand the relation between
physical density and perceived density in another way space syntax
does it. Indeed, the determination of perception of different performances
should be quite simple in relation to cultural aspects, so if our
quantification of performance of density is sufficiently consistent,
we would undirectly make that link through that quantification.\bigskip{}
\bigskip{}
\bigskip{}


\textit{Note : Implicitly this presentation gives the impression that
a strong link exists between space syntax theory and that vision of
density, the top-down approach proposed here seems to exploit the
same internal mechanisms that the bottom-up calculation of spatial
analysis to reveal relations between shape and function in cities.
Need to explore that.}

\newpage{}


\section*{\noun{Ermal Shpuza}, Southern Polytechnic State University\protect \\
Interaction between boundary shape and circulation structure in the
built environment}

Recent research work has been oriented towards the study of the mutual
effect of rules and constraints, in the sense of the relations between
the building shape and the social organization occupying it. These
two elements have totally different time longevity, so we can ask
if it could lead to contradictions between the functionnal aim of
an architecture and its effective use.

\bigskip{}


That lead to the study of two aspects and the links they have : the
boudary shape of the building and the contained circulation. Circulation
system is directly linked to a level of movement, and can be taken
as a local description of floorplates, whereas the boundaries are
more a global description. Such a study can also be done at the urban
scale, by searching the impacts of an imposed shape on internal circulations.

\bigskip{}


We will see here first the pure shape aspect, then the influence of
circulation on shape, and finally the inverse relation.


\subsection*{1 Unique shape approach}

It can be useful to first describe the boundary shape in itself, since
we will interest us later on clustering of shapes.

\bigskip{}


Given a boundary shape, it is possible to extract a polygonal approximation
(which can be exact in the case of a polygonal shape, what is the
most used case for the following studies), and then classify it through
the classification of the polygon. It has been shown ({[}Missing reference{]})
that a polygonal shape can be quite uniquely put in correspondance
with 6 sets of reals numbers, that are, if we note, with $S$ summits
of the polygons, $A_{i}(s)$ the set of depth $i$ adjacent summits
to summit $s$, $S_{i}^{j}=\{d(k,a)^{j}|k\in S,a\in A_{i}(k)\}$,
the particular sets $S_{1}^{1}$, $S_{2}^{1}$, $S_{3}^{1}$, $S_{1}^{2}$,
$S_{2}^{2}$ and $S_{3}^{2}$. Such a classification of polygonal
shapes is the starting point of the following work, since we will
work on unscaled polygonal shapes, i.e. with $\mathcal{P}$ set of
polygonal shape and the equivalence relation on it : $\mathcal{R}:P_{1}\mathcal{R}P_{2}\iff(\exists\alpha\in\mathbb{R}^{*},S_{i}^{j}(P_{1})=\alpha S_{i}^{j}(P_{2}),i=1,2,3,j=1,2)$
, on the quotient set $\nicefrac{\mathcal{P}}{\mathcal{R}}$. On the
following, when we consider polygonal shapes, it will always be on
that set.


\subsection*{2 From circulation to shape : the inside-out approach}

This approach is a modular approach, in the sense that it goes from
inside to outside. The internal space influences the boundary shape.
A shape can be seen as the result of an equilibrium of constraints,
external and internal forces. That approach is the consideration of
the internal forces only, to understand the influence of internal
constraints (for us the internal circulation) on the boundary shapes.
To do that, we concretely consider measures for these two parameters
of the built environment and we plot different classes of polygonal
shapes in the plan among these two measures, in order to try to bring
out clustering patterns between the shapes. The measures are metric
inertia, i. e. compactness and kinetic inertia, i. e. fragmentation
coefficient of the space. {[}\textit{Note : measures not clearly defined}{]}

\bigskip{}


Plotting simple polygonal shapes shows some strong clustering in the
plane, and some shapes appears to be optimal towards both measures.
Adding constraints on the shapes, as holes in polygonal shapes, shows
also strongly localised clustering, what suggest the relevance of
the measures.

\bigskip{}


We can also study the correlation between the two measures by plotting
mazes generated by precise rules on the internal circulation. Seeing
the consequence of these rules on the relation compactness/circulation
gives an idea of the correlation.

\bigskip{}


Finally, one application is plotting on the same graph real configurations
at different scales : building shapes, cities plans, etc. , to classify
the real shapes regarding the clustering already done. No concluding
result have been found yet but this concrete application is our next
goal in the research process.


\subsection*{3 Influence of shape on circulation}

This process is quite the same as the previous one but the measure
are different, we plot here the real shapes according to the real
connectivity and the perceived connectivity (calculated through visual
integration). At a greater scale, this is quite equivalent as studying
the relation between urban shape and street networks. This work is
also still in process, and should lead to a reciprocal confirmation
of the results obtained with the previous approach.

\newpage{}


\section*{\noun{Ulrika Karlsson}, KTH\protect \\
Biotic interferences}

This presentation is more on research in pure design than in spatial
analysis, but is closely linked to it because of the underlying systemic
approach in the design process. It presents a work of integrated design
lead by a multi-disciplinary team at servo Stockholm and KTH, including
researchers from several disciplines such as Design, Architecture,
Ecology of biodiversity, Composition of built materials. The project
was called hydrophile, in relation to its particular aspects that
we will see in the following.


\subsection*{Presentation}

To define the project in itself, which name is ``biotic interference'',
we need to come back to the initial definition of these words : biotic
means related with living organism, and interference should be taken
here as the emergence from sharing by agents of a system. Here the
abstract aim of the project is to create such positives interferences
within a biotic system. It lays on the line of relationship between
technical design (mathematical aspects) and architecture.

\bigskip{}


An ubiquitous idea is the overreaching presence of nature, almost
a symbiosis between all the dirts of nature and human being, as it
can be feeled in the introduction sequence of the swedish film \textit{Melancholia},
or in the artwork \textit{Partially Buried Woodshed} by \noun{Robert
Smithson}.

\bigskip{}


In the 70s, \noun{Banham} built in Los Angeles the first green roof
building, in a exceptionaly innovative way, through the elegant combination
of glass walls with the turf roof. He was one of the first to propose
ecology-oriented analysis of urban systems and environmental designed
architectural projects. He wrote theorical explanations in \cite{banham1984architecture}.
One ambitious aim of the project is to reinvent, to rethink that concept
of green roof, in a innovative approach called the hydrodynamic green
roof.

\bigskip{}


An overview of the project can be seen on figure 3.

\begin{figure}[h]
\includegraphics[scale=0.3]{\lyxdot \lyxdot /Workshop/greenroof/7_hydrophile10new}\caption{Global view of the hydrophile}
\end{figure}



\subsection*{Particular aspects}


\paragraph{Biotope}

A focus was done on the place of biological species in the design
of the project. A biotope map was created (figure 4), including all
animal and vegetal species (as frogs, hooks, conifers, etc) and simulations
on the influence of the project on these species was conducted. The
main goal was to preserve and further encourage biodiversity.

\begin{figure}


\includegraphics[scale=0.4]{\lyxdot \lyxdot /Workshop/greenroof/7_hydrophile5}\caption{Biotope map}


\end{figure}



\paragraph{Technical aspects for the roof}

Since vegetals are implanted on the roof, the thickness of the substrate,
its nature, and the material of the hard roof should have influence
on their development. Several tests around these parameters were done.
One important is the shape of the ground on which the substrate lies,
and therefore real tests were done on miniature versions of the roof.


\paragraph{Rainfall}

Hydrodynamic studies were needed to predict from the map of rainfall
the move of water on the roof. Hydrophobic and hydrophile surface
are used to redirect places to wanted places. The protuberances are
used for irrigation and natural light inside the building.


\subsection*{Conclusion}

The spirit of the project is to integrate biotic processes in archtitecture,
and to have a system globally adaptative to local disturbances, so
it should be integrated vertically and horizontally.

\newpage{}


\section*{\noun{Christian Derix}, AEDAS Architects R\&D\protect \\
Computationnal Design and Advanced Spatial Modeling}


\subsection*{1 Context of the work of Aedas R\&D}

The company proposes to its client powerful applications of computational
design and advanced spatial modeling to design problems, oriented
towards sustainable solutions.

\bigskip{}


The models that are created can be at several different space scales,
but also include people and their interactions between them and with
their environment ; that can be seen as the switch from original space
syntax to computational models for social logic {[}\textit{Note :
In fact, Aedas does in that case nothing more than complex social
system modeling and analysis, but according to their client profile
that are architects and designers, they market it as an evolution
of space syntax}{]}.

\bigskip{}


Examples of outlines of different projects are shown on figure 5 (source
\cite{AedasWeb}).

\begin{figure}[h]
\hfill{}\subfloat[Modeling of solar repartition for towers in Abu-Dhabi]{\includegraphics[scale=0.35]{\lyxdot \lyxdot /Workshop/aedas/Cladding-by-orientation-Abu-Dhabi-UAE-ResearchCladding-by-orientation-697}

}\hfill{}\subfloat[Mapping of cycling flows]{\includegraphics[scale=0.09]{\lyxdot \lyxdot /Workshop/aedas/Cycle-to-Cannes-2011-London-ResearchCycle-to-Cannes-2011-897}}\hfill{}\hfill{}

\hfill{}\subfloat[Global spatial planning solutions]{\includegraphics[scale=0.09]{\lyxdot \lyxdot /Workshop/aedas/Smart-Solutions-for-Spatial-Planning-London-UK-Research}}\hfill{}\hfill{}\caption{Results of Aedas projects}


\end{figure}



\subsection*{2 Examples of recent research projects}


\subsubsection*{Study for masterplan}

It is possible to apply directly syntax to help design ; this example
of project shows how a proposal of masterplan is done and how the
designer can move pieces of it to see the consequences through the
self-organisation of the rest of the plan. The modification can be
done at different scales, from furnitures sometimes to the overall
floor.

\bigskip{}


\textit{Note : The techniques used are called ``agent-based aggregation'',
what seems to be a computational design of possible configuration
(not more precisions since it appears to be confidential for the company).}

\bigskip{}


The figure 6 shows the final result for the Hong Kong polytechnic
University, after the created masterplan has been integrated to the
other sections.

\begin{figure}
\hfill{}\includegraphics[scale=0.16]{\lyxdot \lyxdot /Workshop/aedas/Hong-Kong-Polytechnic-University-Shenzhen-Campus-PRC-1-671}\hfill{}\hfill{}\caption{Hong-Kong southern Polytechnic University}


\end{figure}



\subsubsection*{Distribution of densities}

For the planning of a new business district in China, there was the
need to decide the local densities of activities (in order to then
directly apply it to the local design of buildings). For that, a tool
was created, that allowed the designer to fix some points at a given
density and observe the generated global field of density that resulted
from these imposed values.

\bigskip{}


The method used to extract the interpolating field seems to be not
far from non-parametric estimation (see \cite{tsybakov2004introduction})
: with $n$ given points $(M_{1},...,M_{n})$ in space and the expected
values $(y_{1},...,y_{n})$, the problem is to find a function $f$
such that for all $i$, $f(M_{i})=y_{i}$. That can be done for example
by kernel estimation aggregation.


\subsubsection*{Visibility study}

The construction of the new huge tower on the right bank of the Tamise
in London has raised interrogations about its impact on the visual
landscape of the city. The aim of that project was to model the visual
impacts of that new landmark.

\bigskip{}


To do that, it is possible to calculate by ray-tracing if the tower
is visible from a given point, what was done for a big part of the
city for which the 3D data of building shapes were available. Then
for each point, we can judge if the visible impact is significant,
and also see the total proportion of places for which it has a real
impact.


\subsubsection*{Pedestrian traffic analysis}

For the construction of a new railway station, the locations of entries
for pedestrians had to be decided and a pedestrian flow simulation
model was created.

\bigskip{}


Concretely, it is an easely parametrizable model, for which test could
be done on localization of ``source'' and ``sink'' points for
pedestrian flows. From an external points of view, it is quite similar
to the problem of distribution of densities, although here the interest
is more on the flow quantities resulting from fixed potential points.
But the method to solve the problem is exactly at the opposite, since
for densities it was solved by a top-down calculation, by global mathematical
calculations, and here the model used is a bottom-up approach, since
it simulates the flows through individuals agents that are the pedestrian
themselves.


\subsubsection*{Mapping architectural controversies}

Urban studies are also sociological studies, as this project testify.
Through newspaper articles analysis, it was possible to proceed to
``social mapping'', and identify trending subjects and social clustering
around these key subjects.

\bigskip{}


What is really interesting is to make the parallel between the social
system and the architectural system analysis.


\subsection*{Questions}


\paragraph{Question}

\textit{Do a global comparative knowledge emerge from all these research
projects?}

Yes but not formally, in the sense that no exhaustive list of knowledge
was created in the company. But still, it contributes to the spread
of such knowledge and methods.


\paragraph{Question}

\textit{Was network self-generation already considered in one of the
projects?}

Not at all ; but that seems interesting to consider. Self-generation
following local rules can be a way to proceed to global optimisation
on some propoerties of the network, as nature does in some cases.

\begin{figure}
\includegraphics[scale=0.23]{\lyxdot \lyxdot /Workshop/aedas/Cycle-to-Cannes-2011-London-ResearchCycle-to-Cannes-2011-897}\caption{Project of bicycles flows modeling for prediction of future position
of rented bicycles in London (more detailed view).}


\end{figure}


\newpage{}


\section*{\noun{{\aa}smund Izaki}, AEDAS Architects R\&D\protect \\
Algorithmic aspects of spatial analysis}

That last presentation is a short overview of computer science issues
that occur when doing spatial analysis and investigative modeling.


\subsection*{Complexity of algorithms}

When doing computations, the speed depends on the machine on which
they are done, but what is really important is the intrinsic complexity
of the used algorithm. For example, there exists many ways to sort
a set of number, and the best ones (quick sort or fusion) will have
a mean complexity in $O(n\cdot ln(n))$, what can be assimilated to
a linear time as a function of the size of the set, whereas bad ones
will execute in $O(n^{2})$, what is quadratic and can quickly lead
to impossible calculation times on big data.

\bigskip{}


This aspect is particularly important in spatial analysis because
of the size of the data and the natural complexity of graph exploration
problems, that's why finding ``good'' algorithms for spatial analysis
is necessary.


\subsection*{Difficulty of problems}

When dealing with spatial generation algorithm, some technical problems
appear, as the question of the calculation of visibilities, to obtain
the integration of spatial visibility. It is usually done through
ray-tracing, but that suppose to test for small spatial discretization
if there are edges and if they are open, what can become slow if it
is not done the good way (dynamic programming can help to make the
process quickier).


\subsection*{Examples of applications}

We propose concrete applications that are tools for the architect
or the designer. For example, you have an interactive model calculating
spatial and visual integration, in which you can open/close doors,
add new ones or deleting others. It is a sort of ``design in direct'',
you directly see the consequence of your choices on the properties
of the building. An other example is a 3 dimensionnal configuration
for a building (a school), where you can also modify pieces and see
consequences on internal flows.

\bigskip{}


\textit{Note : main part of the presentation was after that the demonstration
of these softwares.}

\newpage{}\bigskip{}
\bigskip{}



\section*{Conclusion}

\bigskip{}


The architectural solution for a project is a particular response
to the context of the project, a local proposal in space and time,
but it is also a proposition for architecture in general. Architectural
theory builds itself from concrete responses to concrete cases.\bigskip{}
\bigskip{}


What is important to understand is that we need to learn from all
these applications, and from itself the theory should become more
robust. Investigative modeling is still at the beginning but should
become more and more present in architectural projects and in urban
planning tomorrow.

\newpage{}

\bibliographystyle{plain}
\bibliography{ArphMorph}

\end{document}
