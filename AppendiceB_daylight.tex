
\documentclass[english]{article}
\usepackage[T1]{fontenc}
\usepackage[utf8]{luainputenc}
\usepackage{amsmath}
\usepackage{amssymb}
\usepackage{esint}

\makeatletter

\newcommand{\noun}[1]{\textsc{#1}}

\usepackage{bbm}

\usepackage{background}

\usepackage{geometry}[0.5]


%manage background
\SetBgOpacity{0.3}
\SetBgColor{black!20}

\makeatother

\usepackage{babel}
\begin{document}

\title{Basic model for daylight exposition calculation with objective of
urban contexts comparison}


\author{\noun{Juste Raimbault}\\
Chalmers Tekniska Högskola\\
Department of Architecture}
\maketitle
\begin{abstract}
In the frame of a research of objective indicators of different types
in order to compare qualities of given urban contexts, it appeared
that the direct exposition of building windows to daylight (sunbeams)
was quite important for the life quality, since we wanted to apply
our model to districts without particular architectural programs,
current residentials districts. We propose therefore here a simplified
but directly and quickly implementable model for such comparisons,
the aim being more to have an approximate idea if great differences
exists than very precise calculations, because it will be used in
a more general context.
\end{abstract}

\section*{Introduction}

The importance of daylight on the perception of architectural objects,
so therefore of their quality, but also on direct life quality (health
issues, natural need of daylight) are described in \cite{phillips19232004},
and that's why we would like to consider it as an evaluation criteria
for urban districts. We place ourselves in the framework of investigative
modeling that is today strongly developing in architecture (see \cite{ArchMorph})
: we want to use objective calculations and investigations on the
studied objects.

If we consider buildings in themselves, it is possible to estimate
subjectively and objectively the performance of the architecture regarding
daylight treatment, but it is an evaluation of the individual architecture.
More globally, we would like to evaluate the performance of an urban
configuration in that context, that means not if the buildings see
daylight (they always see it of course), but how the sunbeams come
on building facades and the effect of shadows consequences of the
district configuration. A simplified example is if the designer creates
line buildings oriented North-South, it will be significantly more
efficient to get sunbeams on facades than if it was oriented East-West.
Further, there are also the shadows effects between buildings and
natural elements of the site.

In \cite{miguet2002daylight}, the authors study an elaborated model
of daylight calculations, and we have based ourselves on the same
ideas for sunbeam calculations. In the following, we describe the
theorical framework of the model and briefly give clues for quick
concrete implementation.


\section*{Model Description}


\subsection*{Input Data}


\subsubsection*{Sun course}

We want to apply our model to districts located in different places,
but also integrate on short and long time periods, so the input description
of the sun course will depend of the location $L$ (location on earth,
for which only latitude will be influent) and of the time in the year
$T$ (in days).

The space is parametrized in spherical coordinates (adapted to the
calculations), a point in the sky is defined by $(r,\theta,\varphi)$.
It is natural to make the assumption that the sun is located at $r=+\infty$
and that its relative position to every point in the district is the
same, given by the angles $(\theta,\varphi)$. (it stays true as long
as the studied zone remains ``small'', although we didn't quantify
this size, it appears as normal that it is the case for a city district).

The course of the sun will by consequent be represented by a family
of functions
\[
(\mathcal{S}_{L,T})_{L,T}=(t\mapsto(\theta_{L,T}(t),\varphi_{L,T}(t)))_{L,T}
\]


that we will in input roughly approximate by regulars samplings (like
every hour for example) $\mathcal{S}_{L,T}^{(d)}=((\theta_{i,L,T},\varphi_{i,L,T}))_{1\leq i\leq K}$
($K=12$ for 2 hour sampling for example).


\subsubsection*{District configuration}

We need to know the configuration of the buildings, but also their
elevation (shadow effects). A simple way to represent it is just to
give the height function of the position in the district $h(x,y)$,
with $(x,y)\in P\subset\mathbb{R}^{2}$ coordinate within district
bounds relative to an arbitrary origin. We can even by that mean represent
hills or trees (roughly simplified of course). Concretely, the input
data is spatially discretised, and can be for example a GIS raster
data.


\subsubsection*{Set of objective points (windows)}

An other part of the architecture that is needed is the positions
of windows in buildings, that we will consider as a set of $N$ ``objective
points'' : $((x_{i}^{0},y_{i}^{0},z_{i}^{0}))_{1\leq i\leq N}$,
supposed to be compatible with the descriptive function $h$ (the
windows need to be on the wall, so on vertical discontinuities of
$h$). Of course the real windows are more than points but we will
assume the simplification that the windows is enlightened if its center
is (which is the objective point), what shouldn't be a problem because
of the small size of a window compared to a all building.


\subsection*{Calculation of indicators}

Given this simple description of the district configuration, we are
able to calculate indicators of the quantity of direct daylight exposition.

With $\vec{u}_{r}(\theta,\varphi)$ the radial unit vector corresponding
to angles $(\theta,\varphi)$, the approximation that sunbeams come
parallel on all points of our study zone gives directly the equivalence
that $M(x,y,z)$ \textit{is enlightened at time t of day T in L }$\iff\theta_{L,T}(t)>0\textrm{ and }M+\mathbb{R}_{+}\cdot\vec{u}_{r}(\theta_{L,T}(t),\varphi_{L,T}(t))\cap\bigcup_{(x,y)\in P}[(x,y,0)(x,y,h(x,y))]=\emptyset$.

Concretely, the test shouldn't be done on all $P$ but on the projection
of the line for better performance, and the intersection tests can
be done by discretising the segments. It is in fact necessary, since
in 3 dimensions it will be easy to have very close but not intersecting
lines, which intersects in reality, that's why the discretisation
is necessary (if we test among the projection, it will be on its discretisation,
and the given vertical segments will never exactly intersect the line
since we have float values).

Then it is possible to compute the ``enlightening'' functions for
each point :
\[
s_{i}(L,T):t\mapsto\begin{cases}
\begin{array}{cc}
1 & \textrm{ if }(x_{i}^{0},y_{i}^{0},z_{i}^{0})\textrm{ is enlightened at time }t\\
0 & otherwise
\end{array}\end{cases}
\]


so their integration on the day gives the enlightening time $\tau_{i}(L,T)=\intop_{t=0}^{24}s_{i}(L,T)[t]dt$,
measure that we need to normalise if we want to be coherent (compare
absolute enlightening times would have no sense because it depends
directly of $T$ and $L$). In that way we define the normalized enlightening
time $\tilde{\tau_{i}}(L,T)$ by \[\tilde{\tau_{i}}(L,T)=\frac{\tau_{i}(L,T)}{\intop_{t=0}^{24}\mathbbm{1}_{\theta_{L,T}(t)<\frac{\pi}{2}}dt}\]

Then we integrate on the all district by taking the normalized p-norm
of the vector of these values : with $ $$p_{d}\in[1;+\infty[$, we
define $\tilde{\tau}(L,T)=\left\Vert (\tilde{\tau_{i}(L,T)})_{1\leq i\leq N}\right\Vert _{p_{d}}$
(with $p_{d}=1$ it gives the mean and $p_{d}=+\infty$ the max).

To conclude, we just need to consider it on a hole year. With $p_{Y}\in[1;+\infty[$,
the indicator that we will use to approximate and compare the global
daylight expositions is
\[
S(L)=\left\Vert (\tilde{\tau}(L,d))_{1\leq d\leq365}\right\Vert _{p_{Y}}
\]


Since that indicator is normalized by the relative day lengths and
integrated on all district and all year, it should be a good candidate
to represent the ``performance'' of the urban district regarding
the daylight exposition, independently of its location : that tells
if the buildings are agenced to profit at maximum of the light that
is given.

Note that we normalized by an approximation of the length of the day
and not by the global enlightening time, because first it would be
quite complicated to have it as a data, and then it would make not
real sense because the builder can not change anything to that through
design, the geometric configuration of the district has only interaction
with the geometry of the sun course, not with the fact that it is
hidden or not.


\section*{Implementation}

We are currently doing test to implement that model quickly and integrate
it in a more global project. Follows important aspects related to
the implementation.


\paragraph*{Software}

Because we place ourselves in a project which main part consists of
Agent-Based Modeling, we choose the plateform NetLogo (\cite{NetLogo})
for compatibility reasons. In fact it appears to be particularly ergonomic
for that case, because of the efficient GIS extension, and the system
of the world divised in patches that are already the spatial discretisation
we wanted.


\paragraph*{Input data}

Concerning the sun course, as explained before we can take samples
of the position each hour, and the same on the year (each two weeks
for example), so the input can be a simple list of couples.

For the height configuration, we can use GIS raster data as input.
Data can be available from precise Lidar data, a mean to get precise
mappings that have become more popular and efficient quite recently
(see \cite{2004PhDT.......212H} for example), and in some countries
extended coverage has been proceded, at least in great cities (for
Sweden we have available data). These height data are very precise
(50cm) and are particularly adapted to our problem.

Finally, for the windows position, we could input the exact configuration,
but that would require field survey. Instead, we can simplify by approximating
a given number of windows per square meter on facades (will depend
on architecture of course, but we can consider a mean on the district),
then extract hypothetic position from facades shapes (what are themselves
calculated by searching the vertical discontinuities of the height
function), so we don't need more input by doing that way.


\section*{Conclusion}

Although a lot of approximations are done in the model, the approximated
output values can be used for comparison purposes, and that gives
us an objective criteria to evaluate performance of a district regarding
enlightening, that we can integrate in a global comparison of different
districts among criteria of various types.

\bibliographystyle{plain}
\bibliography{\string"/Users/Juste/Documents/Complex Systems/Biblio/BibTeX/global\string",daylight}

\end{document}
