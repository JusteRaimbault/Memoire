\documentclass[english]{article}
\usepackage[T1]{fontenc}
\usepackage[utf8]{luainputenc}
\usepackage{amssymb}
\usepackage{graphicx}

\makeatletter

\newcommand{\noun}[1]{\textsc{#1}}
\newcommand{\lyxdot}{.}


\usepackage{bbm}
\usepackage{background}

\SetBgOpacity{0.3}
\SetBgColor{black!20}

\date{}

\@ifundefined{showcaptionsetup}{}{%
 \PassOptionsToPackage{caption=false}{subfig}}
\usepackage{subfig}
\makeatother

\usepackage{babel}
\begin{document}

\title{Paper B\\
Architectural urbanity : between art and science, how Swedish urban
planning can be considered as particularly performant and valuable}

\maketitle
\vspace{3cm}

\begin{abstract}
We propose to study caracteristics of Swedish urban planning, especially
for the districts built during the Million Homes program. First a
qualitative analysis in general and case studies inform on the
nature of these district and allow us to develop the notion of ``architectural
urbanity''. Then we consolidate the proposed ideas through
quantitative case study by constructing diverse indicators
at the scale of a district and by applying them to the comparison
between two suburbian districts.

\newpage{}
\end{abstract}

\section*{Introduction}

The city cannot (and shoudn't anyway?) be only composed of architectural
exceptions, first of all for technical and pragmatic evident reasons,
then maybe for more complicated internal structure effects. As \noun{Lynch}
has shown in \cite{Lyn60}, the distinction of landmarks and distinctive
edges is essential in the personnal creation of the image of the city,
so in its perception and appreciation, what can be considered as 
influencing the quality of urban life. The apparition of the suburbs
in the second middle of the 20th century directly followed that
logic and took it more dramatically as a general rule, which led
to ``aggregation of unperceptible and disagreable pieces, that we
can difficultly consider as buildings'' (comment on suburbs heard
from an urbanist). It would mean that some parts of urban areas,
especially the suburbs, lack of architectural quality and therefore
of life quality for their inhabitants.

\bigskip{}


However, such a vision remains at a small level of integration and do
not take into account more global descriptors of the urban environment. Indeed
the local architectural or urban qualities of the built environment
can be totally different of more global properties evaluated at the
scale of the district, the entire city or even within system of cities.
\noun{Pumain} (\cite{pumain1997pour}) argued that this more
global vision is necessary to understand, analyse, predict the performance
of urban systems and therefore is necessary for an optimal planning.
This argument is also presented from an other point of view when \noun{Hillier}
propose (\cite{hillier1976space}) spatial and topological analysis
of static urban systems in order to enlight some of their qualities
or defaults.

\bigskip{}


Those arguments are to be developed in the following section, particularly the
ideas developped by \noun{Lars Marcus} in his thesis (\cite{marcus2000architectural}),
where he focuses on the existence of these integrated qualities (what
he calls the ``functional performance'') in a majority of districts.


\section{Swedish Urban planning : subjective qualitative analysis}


\subsection{Architectural urbanity}


\subsubsection{Presentation}

In his thesis, \noun{Marcus'} aim is to show evidences of the existence
of real performance for district planned during the 20th century.
The background is a crisis of functionalism, since the expected emerging
social aspects for district planned in a functionalist way do not
appear today in an obvious way. Such criticism has raised particularly
in Sweden, where the context for projects planned during the Million
Homes program was adopting those views. Looking at reviews on history of urban
planning during the 20th century, a skepticism on possible accumulated
knowledge in that field is quite present in mentalities today, and
that is what \noun{Marcus} wants to contest.

\bigskip{}


The first point of his response is the existence of ``architectural
urbanity'', highlighted as a distinct part of urbanity : ``within
urban planning and design in the 20th century, one can distinguish
a distinct category termed architectural urbanity''. It is not really
clear if this definition requires a success of the top-down planning process behind
most of urbanistic projects, or if this category is a consequence
of commons local aspects, through the emergence of similar effects.
However, the purpose of this first section is not to explain the origin
but to show the reality of this concept. Without going into details,
the methods used to prove this reality are the applications of spatial analysis
methods (see \cite{workshopArchMorph} for further details on this
theory and associated methods ; a written report of the workshop is displayed
in Appendice A). By extracting features of spatial configurations,
we are able to classify the configurations according to quantifications
of parameters of some features. The analogy between the organisation
of a building through its architecture and the noticeable organisation
in these patterns of features is the main argument for calling that
class within planning ``architectural urbanity''.

\bigskip{}


Going further in the second point, he brings proofs that architectural
urbanity implies a minimal level of urban quality. Indeed ``there
are spatial preferences within {[}architectural urbanity{]} which
is possible to link to certain functional performances''. The built
form has both consequences on its meaning and on its function. Therefore
social aspects should be strongly influenced by urban form. The methods
used are the same as for the first point, only brought one step forward, since
he sees these ``quantitative methods as a scientific approach to
the relationship between form and function''.

\bigskip{}


Following this path, we will first try to get a rough feeling what
this notion can mean. This theory will be applied in the second part
with technical means, by looking
for a diversification of performances evaluation approaches.


\subsubsection{Example}


\subsection{A culture of performance ?}


\section{Case study : quantitative comparison}


\subsection{Choice of the subjects}

Proceeding to an evaluation of some evaluation criteria to a single
district would have no sense at all, because if they are not normalised,
reference values are necessarly needed. We propose in the following
to built normalised criteria, but since it would be the purpose of
an entiere study for each to determine the domain of validity and
to assess of their pertinence, they will


\subsection{Selected analysis criteria}


\paragraph{On the variety and the arbitrary in the choice of evaluation criteria}

One could always argue, especially in social complex systems study,
that the chosen evaluation criteria are the reason of the obtained
results and that other criteria, even very close, would have led
to totally opposite conclusions. It is obvious that we can build examples
where for example an arbitrary small change in the degree of an evaluation
norm can lead to arbitrary big changes in the comparison results. 
However, we can suppose that there exists a sort of continuity in
human systems and that the sensitivity of real systems stays small,
in other terms that some near criteria are continuously linked.

\bigskip{}


The choice of evaluation criteria is at the heart of the multi-criteria
decision analysis. An example is \cite{zietsman2006transportation},
where the authors try to propose an heuristic approach for systematic multi-criteria
decision-making, and where the choice of indicators appears crucial
but still arbitrary. They show how ambiguous that problem can be.


\subsubsection{Spatial configuration}

The spatial configuration of a district should have strong influence
on its use by the inhabitants, so on social and economic aspects of
the life in it. Studying the influence of spatial configuration on
human parameters of the city was the original aim of the space syntax
theory when it was first introduced in \cite{hillier1976space}. Since
that, a lot of developments and other applications have been discovered
in that context, creating a sort of informal sub-discipline of urban
planning that we can call ``spatial analysis''. As an example, recent
work on the subject in Sweden focuses on linking attractiveness
of green spaces with their accessibility, which allows to make proposals
on the role of green spaces as in \cite{staahle2010more}. This work
is a further exploration of the general investigation of public open
space that \noun{St}{\aa}\noun{hle} did in his thesis (\cite{staahle2008compact}).\bigskip{}


Of course this approach has raised a lot of debates, especially
on the analytical method used in most of studies. Indeed it is more based
on topological analysis of the relation between subjective spaces
of the configuration than on the real spatial configuration. In \cite{ratti2004urban},
it is shown that the abstract axial line extraction is easily influenced
by arbitrary small space changes.

\begin{figure}
\hfill{}\includegraphics[scale=0.3]{/Users/Juste/Documents/ComplexSystems/SustainableDistrict/Data/Stockholm/Sodermalm/SpatialIntegrationResult}\hfill{}\caption{Implementation of spatial integration calculation (test on Sodermalm
district, Stockholm)}


\end{figure}



\subsubsection{Land use diversity}

The diversity of land use could have an increasing influence on ``urban
quality''. To our knowledge, there is a gap in the litterature
in order to confirm or invalidate that hypothesis.


\subsubsection{Daylight performances}

The role of light is essential in architecture, as Le Corbusier said
in \cite{corbusier1924vers} in this famous quote : ``Our eyes are
made to see forms in light; light and shade reveal {[}...{]} forms'',
or as the work of Tadao Ando devotes to it in the Church of the Light
(Ibaraki Kasugaoka Church in Japan). Daylight has therefore also a
main place in the appreciation of the quality of a dwelling. In \cite{phillips19232004},
\noun{Philips} confirms that ``changing is the heart of daylighting,
perception reacts to change''.

\bigskip{}


It is possible to quantify the performance of a room towards daylighting
by considerations on surface of windows, depth of the room, etc. Here,
we do not consider the small scale of the flat or the building, but
a more global level. We propose in appendice B a simplified model
in order to calculate performance index of a whole district towards daylight.
Such indicator seems to have
never been used systematically during the conception of large scale
urban projects. Of course the question has since long time been solved
locally, e. g. the total renewal of Paris in the 19th century by \noun{Baron
Haussmann }included in its guidelines the need of larger streets,
with calculated ratio between height of buildings and recommended
street width (as he describes in its memories, summed up by \noun{Choay}
in her anthology \cite{choay1965urbanisme}). Nonetheless, a global indicator
depending on geographical position of the district (sun position in
the sky depends strongly on it), topology (hills, differences of floor
heights), natural obstacles (trees) and expressing a normalised value
on a year can be more useful than simple height regulation rules.

\bigskip{}


The construction of the index was based on adaptation at a greater
scale of the basic layer of a daylight calculation model at the
scale of the building proposed in \cite{miguet2002daylight}. In the
following section, since aggregation functions are non necessarily linear.
(in fact the norms used are linear only in the case of the mean and
for positive reals), the order of integration has a meaning. It
appeared more significant to first integrate on a day for each spatial
point (daily performance of the point), then aggregate it on space
(daily performance of district) and finally to evaluate it through
a year.

\bigskip{}



\paragraph{Formal description}

Given a latitude $L$ and a time in the year $T$, we suppose that
the positions of the sun are known (during a day, defined by its spherical coordinates)
: $(\mathcal{S}_{L,T})_{L,T}=(t\in[0;24]\mapsto(\theta_{L,T}(t),\varphi_{L,T}(t)))_{L,T}$, 
as well as the height function of the district $h(x,y)$ on a subset of $\mathbb{R}^{2}$
(we assume the projection has already been done) and the positions
of all windows $((x_{i}^{0},y_{i}^{0},z_{i}^{0}))_{1\leq i\leq N}$.
We can calculate for each window with these data the binaries enlightning
functions (not detailed here) $s_{i}(L,T):[0;24]\rightarrow\{0;1\}$.
Finally, the successives aggregations as explained before give our
index $S(L)$, with $p_{s},p_{Y}\in[1;+\infty]$ parameters for the
norms : \[S(L)=\left\Vert \left(\left\Vert \left(\frac{\intop_{t=0}^{24}s_{i}(L,d)[t]dt}{\intop_{t=0}^{24}\mathbbm{1}_{\theta_{L,T}(t)<\frac{\pi}{2}}dt}\right)_{1\leq i\leq N}\right\Vert _{p_{s}}\right)_{1\leq d\leq365}\right\Vert _{p_{Y}} \]The
point particularly relevant in this indicator is that it expresses
the capacity of a district to use the given daylight, thanks to the
normalization by the lighting time in a day ; in winter in Sweden,
enlighting times are short, but that does not mean that the district
is not designed in an efficient way. This point raises the question
of the validity field of the indicator : as the latitudes get closer to huge (?)
North or South (just Kiruna in north Sweden could already pose a problem),
the indicator will become very sensitive to the local configuration because of the small
values of enlightning time and might lose its meaning (it should
be possible to build masterplans leading to strong bifurcation phenomena)
Investigation on validity domain through linking to perceived impressions
and cultural expectations and possibly diversification of their expression
depending on human parameters, could be the object of further study.
Still, we will use that this indicator here as an approximative indicator.

\bigskip{}


The figure 4 shows the implementation through sunbeams trajectory
calculation.

\begin{figure}
\hfill{}\includegraphics[scale=0.26]{/Users/Juste/Documents/ComplexSystems/SustainableDistrict/Results/DistrictAnalysis/light}\hfill{}\hfill{}\caption{Implementation of daylight indicator}


\end{figure}



\subsubsection{Public transportation performances}

The last criteria we choose is the performances of the public transportation
network ; it is in another field than the three we already have and
appears also as crucial in the life of the city. Without getting into
the debate of the structuring aspect of the transportation network
or of its interactions with the other aspects of urban systems, we
will accept the reasonable possibility that the relationship between the
quality of the transportation network (according to various criteria
as robustness, connectivity, speed, etc) and the ``quality'' of
the district (which is of course not really defined) is an increasing
function.

\bigskip{}


The intermediate scale for which we give indicators is not appropriated
for complex network analysis such as centrality measures (see \cite{crucitti2006centrality}
on physical street network), clustering analysis (see \cite{jiang2004topological}
on named street network), or real robustness evaluation (see \cite{RePEc:eee:transa:v:44:y:2010:i:5:p:323-336}
on large scale road network weighted by real traffic). However the promising
recent developments in complex networks theory (in particular by application
of results on complex networks in physics to the study social networks)
should invite us to find a way to consider networks at this smaller
scale and try to apply these methods ; that could be the object of
future work.

\bigskip{}


For our district evaluation, we will use two very simple but essential
indexes : transportation time and network relative speed. Formally,
we suppose having the abstract representation of pedestrian network
as an euclidian graph $(V_{p},E_{p})$ with $V$ finite part of the
euclidian plane, and the same for the public transportation network
$(V_{t},E_{t})$. Each building will correspond to a node in $V_{p}$,
so the buildings can be seen as $B\subset V_{p}$. Transportation
network is accessible by foot, i. e. for all $v\in V_{p}$, there
exists at least a path to a vertex $v'\in V_{p}\cap V_{t}$. Then
we are able to define the shortest path to transportation for all
$b\in B$, and the nearest station $s(b)$. To simplify, there exists
a ``target station'' $T$ in transportation network which is concretly
the town center. We note $d_{v,v'}$ the length in networks between
vertices $v$ and $v'$. With mean speeds pedestrian speeds $v_{p}$
and transportation speed $v_{t}$ (we neglige waiting time, which
can be integrated in speed if needed), we define the first indicator
for the network (homogeneous to a time), with $p_{t}\in[1;+\infty[$
:
\[
\tau=\left\Vert \left(d_{b,s(b)}\cdot v_{p}+d_{s(b),T}\cdot v_{t}\right)_{b\in B}\right\Vert _{p_{t}}
\]


The other network performance we propose is the relative ``speed'',
i.e. the quantification of how the network is able to convey directly
the passengers to its destination. It appears to be an essential metric in real network
studies as it is explained in \cite{banos2012towards}. With the same
notations as previously, we will calculate it only on the pedestrian
network, since at our scale the transportation network adapts itself
to local constraints, quantifying its performance that (laquelle?) way would have
no sense. The undimensioned ``relative speed'' integrated on all
trips is defined as follows, with $p_{s}\in[1;+\infty[$,
\[
\sigma=\left\Vert \left(\frac{d_{b,s(b)}}{\sqrt{(x_{b}-x_{s(b)})^{2}+(y_{b}-y_{s(b)})^{2}}}\right)_{b\in B}\right\Vert _{p_{s}}
\]



\subsection{Quantitative comparison}


\subsubsection{Extraction of data}

The GIS data for building shapes of the Bergsjön district in Göteborg
are available from internal database, but we do not adapt the
shapefiles for paths and roads (spatial analysis) or for the transportation
network. Furthermore, data for the French district could not be accessed to. 
Indeed, they are not freely provided by the French geographic national institute ("Institut Géographique National",IGN). 
That's why we proceed to the data extraction on our own
in a Desktop GIS Software (\cite{QGIS_software}), for all different
layers : one layer for buildings used for landuse diversity, daylight
performances calculation and public transportation performances (generation
of origin/destination matrix), one for pedestrian paths for spatial
configuration analysis and a third layer for public transportation
(simplified to the tram lines, since the bus lines are not significant
in both districts). Technically, an export to ESRI shapefile format
makes the analysis by the Netlogo code possible. We can see
in figure ?? the image of the layers that are used for both districts.

\begin{figure}


\subfloat[Les Minguettes, Vénissieux, France]{

\includegraphics[scale=0.24]{images/PaperB/dataMinguette}

}\subfloat[Bergsjön, Göteborg, Sweden]{

\includegraphics[scale=0.26]{images/PaperB/dataBergsjonPart}}\caption{GIS Data used for district analysis}


\end{figure}



\subsubsection{Results}

For both districts we calculate the indicators presented above
for different values of the parameters for the p-norms. All the results
are summed up in table 1 for norms as means. The curves show comparison
of results with different values for the parameters.


\paragraph{Interpretation and discussion}


\section*{Conclusion}

\newpage{}

\bibliographystyle{plain}
\bibliography{/Users/Juste/Documents/ComplexSystems/Biblio/BibTeX/global}

\end{document}
