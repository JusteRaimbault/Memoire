\documentclass[english]{article}
\usepackage[T1]{fontenc}
\usepackage[utf8]{luainputenc}
\usepackage{amstext}
\usepackage{amssymb}

\makeatletter

\newcommand{\noun}[1]{\textsc{#1}}

\usepackage{background}
\usepackage{multicol}
\usepackage{bbm}

%\usepackage{geometry}[0.8]


%manage background
\SetBgOpacity{0.3}
\SetBgColor{black!20}

\makeatother

\usepackage{babel}
\begin{document}

\title{Application of agent-based modeling to the consolidation of a multi-criteria
evaluation framework for building refurbishment}
\maketitle
\begin{abstract}
Write at the end
\end{abstract}

\section*{Introduction}

As we explained in \cite{RaimbaultReBoDescription0412}, the main
part of our project on ReBo and L{\aa}ngängen case is the conception
and implementation of an agent-based model at the scale of the district,
which will take into account several aspects, from economic field
to socio-cultural issues. This requires a heterogeneous model, which
also need to be multi-scaled, since the data we will use
are themselves displayed on various scales (macro scale for economic data
such as unemployment, micro scale for the sociological data coming
from individual questionnaires submitted recently to households of
the district in the frame of a linked project). It is well known that,
beyond epistemological issues that are also a great part of the problem
according to \cite{michel2004formalisme}, the technical aspect of
building such models is a huge obstacle. An appropriate methodology,
proposed in \cite{Duboz:phd}, is to construct the model step by step,
beginning with the essential in a simple way. Even following this,
it appears that the first step is a great challenge in itself.

We apply this methodology in consequence ; as it seems, it is the
first aspect necessary to produce a credible and working model
is the economic one. This has led to the building a simple economic model
to which we could add the other aspects later.

After a review of existing applications of Agent-based Modeling to
urban questions, synthetized in \cite{ABMReview07,ABMPers12} , it seems that
economic models of urban systems never were the purpose of
pure economic research. Indeed, the economic field seems to have difficulties
to accept the synergetics approach to their problems as it is argued
in \cite{farmer2009economy}, but that there already exists promising
economic agent-based models applied to geography. The SimPop model,
developped for ten years by \noun{Sanders} \& \textit{al}. , presented
in \cite{sanders1997simpop}, appears to be the nearest model to what we
aim to do, but at a greater scale, remaining however adaptable at
our smaller scale.


\section{Background economic model}


\subsection{Scales and data}

The real data we dispose of are of two types : macro yearly data such
as unemployment rates at the scale of Gothenburg city or its part
containing L{\aa}ngängen (Kvillbäcken), obtained from local statistic
office, and micro data at the scale of the building or the household,
for example the rent values obtained from the companies renting the
buildings (annual reports), or incomes of households and subjective
considerations about the quality of life from the questionnaires recently
submitted.

That implies a double-scaled model : the evolution of agents and interactions
between them will take place at the smallest level, which will be
the flat and the household, with a small temporal scale as two weeks
for example, whereas the district will also be considered from a global
point of view to integrate the macro data, which will then be seen
as exogeneous fixed time-series, on which the local behaviours will
have no influence, but which the integration of local values will
tend to reproduce in the better way. This hypothesis seems to be reasonable
if people work mainly outside the district, so a local perturbation
of the economic configuration will have a small impact on the unemployment
(for example, if everybody works at the local supermarket and it burns,
this hypothesis will not be accepted, whereas if the main part works outside,
the fire will not have direct impact on unemployment - it will have of
course on other aspect such as life quality).


\subsection{Agents}

For the basic model, the agents are households and flats.

A household $h\in\mathcal{H}(t)$, with $\mathcal{H}(t)$ set of households
at time $t$, is described by the number of people in it, the incomes
status (with the convention that non-actives can't have income and
unemployed have null income), the level of competence (approximately
level of studies) of each active in it, the years of experience in
the current job for each active, a consumption rate $c_{h}$ which
describes approximately the propensity to consume and the occupied
flat $f\in\mathcal{F}(t)$ ($\mathcal{F}(t)$ set of flats), so he
have for all (???) $t\geq0$,
\[
\mathcal{H}(t)\subset\left[\left|1;p_{max}\right|\right]\times\bigcup_{k\leq p_{max}}([0;I_{max}]\times[0;s_{max}]\times[0;e_{max}])^{k}\times[0;1]\times p_{\mathcal{F\rightarrow\mathbb{N}}}(\mathcal{F}(t))
\]


where $p_{max}$ is the maximal number of people in a household, $I_{max}$
(resp. $s_{max}$, $e_{max}$) the maximal income in SEK (resp. level
of studies in years, resp. experience in current work) for actives,
and $p_{\mathcal{F}\rightarrow\mathbb{N}}$ a projection of $\mathcal{F}$
on $\mathbb{N}$ given by an arbitrary numerotation.

For the set of flats, we considere more simply the number of rooms
and the rent, since the location is not of interest yet (the building
determines the rent per square meter, but it does not matter if we take
the rent as given). We also take into account the surface is in first approximation
directly linearly linked with the number of rooms. That means that
for all $t\geq0$, $\mathcal{F}(t)\subset\left[\left|1;R_{max}\right|\right]\times]0;r_{max}]$,
with $R_{max}$ maximal number of rooms, $r_{max}$ maximal rent.


\subsection{Evolution and interactions}

The economic situation of households evolves according to the global
economic data and the evolution of flat rents, whereas the evolution
of flat rents is function of themselves and the mean economic value
of the district.

An iteration step of the model, corresponding to the small time step,
is globally the following :
\begin{itemize}
\item Update work situation ; some people loose their jobs, other find one,
in order to fit the current unemployment macro-value. Social help
attribution and work experiences are also updated.
\item Calculate economic balances for each houshold, taking into account
all expenditures (rent, taxes, consumption).
\item Households in great difficulty leave the district (seen as too expensive
for them), whereas new inhabitants can come.
\item Update rents for each building.
\end{itemize}
We describe the rules in details in the following.


\subsubsection{Initialisation}

The initialisation is done both randomly and following external data.
For example, the buildings follow their real GIS shapes. The exact
way to initialise will not be described and can be seen in source
code if needed. The important point is the one of the internal coherence
of the implementation, so the variables of households are initialised
following the update rules explained in the following section, in order to
have no transitionnary state before the ``real behaviour'' of the
model (that means that the model does directly what it should do and
that there is no side-effects due to wrong initialisation, it seems that
we are following a situation that has already evolved before).


\subsubsection{Update work situations}


\paragraph{Employment}

For a household $h$, we can define by projection and count the
number $u_{h}$ of people unemployed and the number $a_{h}$ of actives
in it, so the total number of unemployed is $U(t)=\sum_{h\in\mathcal{H}(t)}u_{h}$
and the unemployment rate is $u(t)=\frac{\sum_{h\in\mathcal{H}(t)}u_{h}}{\sum_{h\in\mathcal{H}(t)}a_{h}}$.

The global tendance of $u$ should verify $<u(t_{n})>=u_{n}$, where
$u_{n}$ is the fixed external time-serie. If $N_{u}(t_{n+1})$ is
the number of people loosing their jobs and $N_{e}(t_{n+1})$ the
number of people finding a job, we have approximatively $N_{u}(t_{n+1})-N_{e}(t_{n+1})=A(t_{n})\cdot(u(t_{n+1})-u(t_{n}))$
(taking the number of active $A(t)$ as constant at this time scale,
that suppose that emigration and immigration rates are relatively
small). New employed are directly linked to a global economic tendancy,
that can be also represented by a time serie $J_{n}$ (which represents
the quantity of job opportunities), in order to write $N_{e}(t_{n+1})\sim\mathcal{N}(J_{n+1},\sigma_{J})$
(the variance is a fixed parameter). Therefore we can calculate the
number of people loosing their jobs also as a random function ($\sigma_{u}$
parameter) :

\[
N_{u}(t_{n+1})\sim\mathcal{N}(\mathcal{N}(J_{n+1},\sigma_{J})+A(t_{n})\cdot(u_{n+1}-u(t_{n})),\sigma_{u})
\]


That satisfies our constraint on the mean value, we can see it by taking
the mean on the above formula.%
\footnote{Note that to be correct, we should note $N_{e}(t_{n+1})\sim_{law}\mathcal{N}(J_{n+1},\sigma_{J})$
and $N_{u}(t_{n+1})\sim_{\textrm{cond. to }N_{e}=n_{e}}\mathcal{N}(n_{e}+A(t_{n})\cdot(u_{n+1}-u(t_{n})),\sigma_{u})$,
and the calculation of mean is done conditionally : $\mathbb{E}(N_{u}(t_{n+1}))=\mathbb{E}(\mathbb{E}(N_{u}(t_{n+1})|N_{e}))=\mathbb{E}(N_{e}+A(t_{n})\cdot(u_{n+1}-u(t_{n})))=J_{n+1}+A(t_{n})\cdot(u_{n+1}-u(t_{n}))$,
what gives well $<u(t_{n+1})>=u_{n+1}$. %
}

A new employee fixes (?? répare son salaire?) his income such that $I_{i}(t_{n+1})=max(I_{min},\mathcal{N}(m_{I}-\sigma_{I}+k_{I}\cdot s_{i}\cdot\sigma_{I},\sigma_{I}))$
where $I_{min}$ is a minimal income, $m_{I}$ a mean value, $\sigma_{I}$
a standard deviation, $k_{I}$ a parameter representing the importance
of studies in income attribution (if $k_{I}=1$, mean income will
correspond to 1 year of studies. Such a relation is of course reducing
but we will take that in first approximation).


\paragraph{Experience}

After that, for each household $h$, if $(e_{i}(t))$ are the experiences
and $(I_{i}(t))$ the incomes, $\tau$ the value of the time step,
we set $e_{i}(t_{n+1})= \mathbbm{1}_{I_{i}(t_{n+1})=0}\cdot e_{i}(t_{n})+ \tau$
and if the income has not already be defined through a new employment,
$I_{i}(t_{n+1}) = \mathbbm{1}_{e_{i}(t_{n+1}) \textrm{ mod } \tau_{prom} = 0}\cdot (I_(t_{n})+I_{step})$
where $\tau_{prom}$ is the duration to be promoted (necessary multiple
of $\tau$) and $I_{step}$ the corresponding income variation.


\paragraph{Social Help}

At each step, a given subset $\mathcal{S}(t)$ of $\mathcal{H}$ can
benefit of social help. We define the primary reduced balance by $b_{r}(h)=(\sum I_{i}-r(f(h)))/p(h)$,
with $f$ the occupied flat, $r$ its rent function and $p$ the people
number. Following a simple threshold criterium, the households in $\mathcal{S}$
are the $S_{max}$ elements in $\{h\in\mathcal{H}|b_{r}(h)<t_{S}\}$
with the lowest $b_{r}$.


\subsubsection{Economic balances}

Then the global economic balance is calculated for each household.
It is defined, taking $b_{p}(h)=(1-t-c_{h}\cdot p_{h})\sum I_{i}-r(f(h))$,
by $b(h)=max(b_{p}(h),\mathbbm{1}_{h\in \mathcal{S}} \cdot min(I_{S}^{max},b_{p}(h)+I_{S}))$,
with $t$ global taxe rate (we consider the tax to be linear in a first approximation
; a threshold function could be more appropriate), $c_{h}$ the propension
to consume (per person) of the household, $p_{h}$ the number of people,
$I_{S}$ the max amount of social help and $I_{S}^{max}$ the maximal
value of balance after getting social help


\subsubsection{Population movements}

According to the value of their economic balance, people can then
leave the district if it has become too expensive for them. That is
modeled simply by a threshold rule again, households with $b(h)<t_{d}$
are deleted from $\mathcal{H}$ at this step.

Then new inhabitants can immigrate, if their mean income is the current mean
income and if other parameters are chosen following the current initialisation
rule (can be at random or depending of $u_{n}$ for unemployment for
example). Their number is limited by the number of free flats $Card(\{f\in\mathcal{F}|\forall h\in\mathcal{H},f(h)\neq f\})$
and a maximal number $N_{n}$.


\subsubsection{Update rents values}

It seems logical to suppose that rents are updated according to the
previous values and mean of rents (to have an auto-regulation) and
to the global ``economic value'' of the district, which we can see
as a direct function of incomes. Furthermore, a strong legal disposition
particular to Sweden fixes a threshold that rents can not exceed to
regulate the rents. To follow these constraints, we propose a rent
update for flat $i$ of the form

\[
\begin{array}{c}
r_{i}(t_{n+1})\sim_{law}\mathcal{N}(min(r_{max}\cdot n_{r}(i)\cdot s_{0},(1+\frac{<b(h)>_{\mathcal{H}}(t)-b_{ref}}{b_{norm}})\cdot r_{i}(t_{n})\\
+K_{r}\cdot(<r_{i}>_{\mathcal{F}}(t)-r_{i}(t_{n}))),\sigma_{r})
\end{array}
\]


where $\sigma_{r}$ is the variance parameter, $r_{max}$ the maximal
rent per square meter fixed by the law, $n_{r}(i)$ the bumber of
rooms in flat, $s_{0}$ the mean surface of a room (the random fluctuations
are translated through the randomness of rent), $b_{ref}$ and $b_{norm}$
two parameters quantifying the influence of economic status on rents,
$K_{r}$ the auto-regulation coefficient.

The term of auto-regulation tends to create a uniform distribution
of rents on long times, because it brings all rents to the mean value.
This is why we will for long time executions set the parameter $K_{r}$ to
zero.


\subsection{List of parameters}

The right boundaries of descriptive variables for households and flats
are parameters in themselves, but since we need to take them arbitrary
big to always stay in the definition domain because they do not appear
in formula, they have no influence on the model behaviour, so we do not
consider them.

Follows the list of essential parameters.
\begin{itemize}
\item $\tau$ the time step corresponding to the real value of one iteration
\item $J_{n}$, $u_{n}$ the fixed time-series corresponding more to data
than to parameters, and their corresponding sample time step $\tau_{ext}$
\item $I_{min}$ the minimal income, $m_{I}$ the mean income (that we could
also take as a time serie later, to represent for example the external
consequence of inflation on incomes), $\sigma_{I}$ the corresponding
deviation, $k_{I}$ the coefficient determining the importance of
competence level
\item $\tau_{prom}$ the experience time to get promoted, $I_{step}$ the
corresponding income increasing
\item $S_{max}$ the maximal number of households that can get simulteanously
social help, $t_{S}$ the social help attribution treshold, $I_{S}$
the maximal amount of money a household can get from social help,
$I_{S}^{max}$ the maximal final balance for people getting social
help
\item $r_{max}$, $s_{0}$, $b_{ref}$, $b_{norm}$, $K_{r}$ the rent parameters
\item $t_{d}$ die treshold and $N_{n}$ maximal number of immigrants per
step
\end{itemize}

\section{Integration of multiple aspects}

The next step of the model construction process iss to consider the
first layer as validated, and to add new layers of agents and variables
in order to create a heterogeneous integrated model. The validation
step was made through model exploration and sensitivity analysis,
that we describe further in the results section.


\subsection*{Choice of extending aspects}

The main objective is to diversify the model in the direction of
socio-cultural aspect, since the purpose of ReBo is to hightlight
the importance of these multiple aspects in the refurbishment process.


\subsection*{Description of the integration}


\section{Concrete results : refurbishment simulation}


\subsection{Implementation and calibration}


\subsubsection{Implementation issues}

The model was implemented with the NetLogo software (\cite{NetLogo}),
which is designed for agent-based modeling.


\subsubsection{Calibration process}

The aim of the model is to make predictions based on real data, so
after reducing the number of unknown (mostly because abstract)
parameters in the model, we designed a particular calibration process,
in order to automatically find one of the best set of values for the
parameters that makes the model ``fit the reality'' (according to
some objective functions).

The technical details of the elaboration process (which includes exploration
and experiments) are developped in Appendice and we will just sum
up here briefly the method used.


\subsection{Case study : district L{\aa}ngängen}


\subsubsection{Data collection and integration}


\subsection{Simulating possible refurbishment scenarii}


\section{Discussion and perspectives}


\subsection*{Obtained results}


\subsection*{Limitations of agent-based modeling}


\subsection*{Consolidation of the original framework?}


\subsection*{Perspectives of developments}

Next steps in the project can be directed towards the application
to other concrete case, that could 


\section*{Conclusion}

\bibliographystyle{plain}
\bibliography{/Users/Juste/Documents/ComplexSystems/Biblio/BibTeX/global}

\end{document}
